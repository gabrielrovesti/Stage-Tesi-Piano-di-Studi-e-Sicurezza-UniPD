%----------------------------------------------------------------------------------------
%   USEFUL COMMANDS
%----------------------------------------------------------------------------------------

\newcommand{\dipartimento}{Dipartimento di Matematica ``Tullio Levi-Civita''}

%----------------------------------------------------------------------------------------
% 	USER DATA
%----------------------------------------------------------------------------------------

% Data di approvazione del piano da parte del tutor interno; nel formato GG Mese AAAA
% compilare inserendo al posto di GG 2 cifre per il giorno, e al posto di 
% AAAA 4 cifre per l'anno
\newcommand{\dataApprovazione}{Data}

% Dati dello Studente
\newcommand{\nomeStudente}{Gabriel}
\newcommand{\cognomeStudente}{Rovesti}
\newcommand{\matricolaStudente}{2009088}
\newcommand{\emailStudente}{gabriel.rovesti@studenti.unipd.it}
\newcommand{\telStudente}{+ 39 346 68 89 789}

% Dati del Tutor Aziendale
\newcommand{\nomeTutorAziendale}{Fabio}
\newcommand{\cognomeTutorAziendale}{Pallaro}
\newcommand{\emailTutorAziendale}{f.pallaro@synclab.it}
\newcommand{\telTutorAziendale}{+ 39 333 13 68 8500}
\newcommand{\ruoloTutorAziendale}{}

% Dati dell'Azienda
\newcommand{\ragioneSocAzienda}{Sync Lab S.r.l}
\newcommand{\indirizzoAzienda}{Galleria Spagna, 28, Padova (PD)}
\newcommand{\sitoAzienda}{https://www.synclab.it/}
\newcommand{\emailAzienda}{info@synclab.it}
\newcommand{\partitaIVAAzienda}{P.IVA 07952560634}

% Dati del Tutor Interno (Docente)
\newcommand{\titoloTutorInterno}{Prof.ssa}
\newcommand{\nomeTutorInterno}{Ombretta}
\newcommand{\cognomeTutorInterno}{Gaggi}

\newcommand{\prospettoSettimanale}{
     % Personalizzare indicando in lista, i vari task settimana per settimana
     % sostituire a XX il totale ore della settimana
     \begin{enumerate}
        \item \textbf{Settimana 1-2 (40 ore)} - Introduzione al progetto di stage e presentazione degli obiettivi da raggiungere. Formazione di base sulle tecnologie blockchain e self-sovereign identity. 
        \item \textbf{Settimana 3-4 (40 ore)} - Blockchain: studio del funzionamento della blockchain, concetto di wallet e firma asimmetrica delle transazioni, validazione e mining dei blocchi, tipologie di consenso, Smart contract e linguaggio Solidity, limiti e scalabilità della tecnologia blockchain. Studio della creazione di token su blockchain.
        \item \textbf{Settimana 5-6 (40 ore)} - Self-sovereign identity: studio del concetto di SSI, protocolli e tecnologie per la gestione delle identità digitali, analisi delle caratteristiche e delle criticità di ciascuna di esse. Studio delle vulnerabilità principali in Solidity.
        \item \textbf{Settimana 7-8 (40 ore)} - Crittografia e protocolli utizzati nella blockchain: approfondimento delle tecniche utilizzate per garantire la sicurezza e la privacy delle informazioni personali nell'ambito della blockchain e della SSI, ruolo della crittografia nell'immutabilità della blockchain.  
        \newpage
        \item \textbf{Settimana 9-10 (40 ore)} - Casi d'uso reali di self-sovereign identity: individuazione di casi d'uso reali di SSI e presentazione di esempi di applicazioni attuali. Studio dei consorzi internazionali coinvolti e dei finanziamenti europei legati alla self-sovereign identity.
        \item \textbf{Settimana 11-12 (40 ore)} - Zero Knowledge Proof: studio di cos'è e come potrebbe servire per la SSI, analisi delle applicazioni reali e delle sfide tecniche da affrontare.
        \item \textbf{Settimana 13-14 (40 ore)} - Approfondimento degli aspetti di sicurezza della blockchain: studio delle vulnerabilità principali in Solidity, tecniche e strumenti per la sicurezza degli smart contract.
        \item \textbf{Settimana 15-16 (40 ore)} - Possibili scenari futuri di applicazione della blockchain e della SSI: studio di un possibile scenario futuro di applicabilità, discutendo problemi e sfide correlati. 
    \end{enumerate}


}

% Indicare il totale complessivo (deve essere compreso tra le 300 e le 320 ore)
\newcommand{\totaleOre}{320}

\newcommand{\obiettiviObbligatori}{
	 \item \underline{\textit{O01}}: Descrivere i concetti di base della blockchain, tra cui la sua architettura, i nodi della rete, la criptografia e il consenso distribuito;
	 \item \underline{\textit{O02}}: Analizzare il concetto di Smart contract e il linguaggio Solidity, con particolare attenzione alle vulnerabilità principali e alle tecniche per evitare errori di programmazione;
	 \item \underline{\textit{O03}}: Approfondire il funzionamento della firma asimmetrica delle transazioni su catena e la validazione dei blocchi, studiando le tipologie di consenso e le catene più conosciute;
	 \item \underline{\textit{O04}}: Studiare le tecniche di crittografia utilizzate per garantire la sicurezza e la privacy delle informazioni personali nell'ambito della Self-Sovereign Identity (SSI) e dei protocolli per la gestione delle identità digitali;
     \item \underline{\textit{O05}}: Individuare casi d'uso reali per la SSI, analizzando i consorzi internazionali coinvolti in ambito di ricerca e i finanziamenti europei;
     \item \underline{\textit{O06}}: Discutere le sfide e i problemi legati alla SSI, studiando un possibile scenario futuro di applicabilità e basato su Zero Knowledge Proof (ZKP).
}

\newcommand{\obiettiviDesiderabili}{
	 \item \underline{\textit{D01}}: Analizzare la scalabilità e i limiti della tecnologia blockchain;
	 \item \underline{\textit{D02}}: Analizzare le criticità e le sfide nella gestione dell'identità digitale, come la perdita o la violazione dei dati, la mancanza di standard e la scarsa adozione da parte degli utenti;
	 \item \underline{\textit{D03}}: Esplorare le tecnologie di smart contract e di blockchain programmabile, per esempio Ethereum, e il loro possibile utilizzo in un sistema SSI;
	 \item \underline{\textit{D04}}: Approfondimento delle tecniche di tokenizzazione e creazione di token su blockchain.
}

\newcommand{\obiettiviFacoltativi}{
	 \item \underline{\textit{F01}}: Approfondire gli aspetti tecnici relativi alla blockchain, analizzando le vulnerabilità principali e le tecniche per mitigarle;
	 \item \underline{\textit{F02}}: Analisi di progetti blockchain già esistenti, individuando criticità e punti di forza;
	 \item \underline{\textit{F03}}: Studio di altre tecnologie emergenti nel campo della gestione delle identità digitali, come ad esempio la tecnologia DID (Decentralized Identifiers).
}