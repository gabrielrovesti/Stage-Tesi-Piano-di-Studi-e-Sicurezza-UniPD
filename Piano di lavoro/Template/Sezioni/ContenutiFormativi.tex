\section*{Contenuti formativi previsti}
% Personalizzare indicando le tecnologie e gli ambiti di interesse dello stage
Durante questo progetto di stage lo studente avrà occasione di approfondire le sue conoscenze in ambito blockchain e self-sovereign identity,
come indicato di seguito in dettaglio.
\begin{itemize}
    \item \textbf{Concetti di base blockchain}
    \begin{itemize}
        \item Studio del funzionamento della blockchain;
        \item Concetto di wallet e funzionamento firma asimmetrica delle transazioni su catena;
        \item Validazione e mining dei blocchi;
        \item Tipologie di Consenso e studio delle catene più conosciute;
        \item Concetto di Smart contract e linguaggio Solidity;
        \item Scalabilità e limiti della tecnologia blockchain;
        \item Tokenizzazione e creazione di token su blockchain;
        \item Concetto di immutabilità nella blockchain e il ruolo della crittografia;
        \item Vulnerabilità principali in Solidity (e.g. reentrancy, integer overflow/underflow, etc.).
    \end{itemize}
    \item \textbf{Self-Sovereign Identity e Zero Knowledge Proof}
    \begin{itemize}
        \item Studio del concetto di self-sovereign identity (SSI);
        \item Descrizione di protocolli e tecnologie per la gestione delle identità digitali;
        \item Implementazione mediante blockchain;
        \item Approfondimento delle tecniche di crittografia utilizzate per garantire la sicurezza e la privacy delle informazioni personali nell'ambito della SSI;
        \item Casi d'uso reali individuati;
        \item Consorzi internazionali coinvolti e finanziamenti europei;
        \item Studio di un possibile scenario futuro di applicabilità, discutendo problemi e sfide correlati;
        \item Zero Knowledge Proof: cos'è e come potrebbe servire per la SSI.
    \end{itemize}
\end{itemize}


