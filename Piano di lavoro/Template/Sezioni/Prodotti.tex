%----------------------------------------------------------------------------------------
%	DESCRIPTION OF THE PRODUCTS THAT ARE BEING EXPECTED FROM THE STAGE
%----------------------------------------------------------------------------------------
\section*{Prodotti attesi}
% Personalizzare definendo i prodotti attesi (facoltativo)
Lo studente dovrà produrre una relazione scritta che riassume lo studio condotto nel periodo invididuato,
fornendo una panoramica completa e approfondita delle tecnologie oggetto di studio (analizzando le caratteristiche, le potenzialità e le criticità di ciascuna di esse).
Questa, in particolare, illustra i seguenti punti.
\begin{enumerate}
    \item Introduzione \\
    Descrizione del contesto in cui si inserisce il progetto di stage e del problema affrontato, sulla base dello studio svolto e della realtà aziendale presente.
    In questa sezione sarà inclusa un'analisi del contesto di lavoro, spiegando la motivazione del progetto e dei suoi obiettivi ed impatti.

    \item Analisi delle tecnologie \\
    Descrizione delle tecnologie oggetto di studio, con analisi delle caratteristiche, delle potenzialità e delle criticità di ciascuna di esse. Qui verranno inoltre riportate eventuali vulnerabilità di sicurezza. 
    Per ciascuna di queste, saranno individuati opportuni casi d'uso reali. Le sottosezioni dedicate forniranno un quadro completo dell'utilizzo delle tecnologie, commentando nel dettaglio le caratteristiche utili 
    per risolvere il problema affrontato.

    \newpage

    \item Scenario di applicabilità \\
    Studio e descrizione di un possibile scenario di applicazione delle tecnologie oggetto di studio, con analisi dei vantaggi e degli svantaggi rispetto a soluzioni alternative.
    Questa ha l'obiettivo di individuare campi di applicazione, sulla base dei vari consorzi internazionali interessati e analizzando casi d'uso reali. 

    \medskip
    
    In particolare, verrà analizzato il caso di applicazione della SSI, con particolare riferimento alla tecnologia Zero Knowledge Proof.
    L'analisi sarà svolta criticamente, contestualizzata e con riferimento a soluzioni alternative, individuando punti di forza e limiti presenti.
    

    \item Conclusioni e sviluppi futuri.
    Riassunto dei risultati ottenuti, delle conclusioni raggiunte e delle eventuali prospettive future di sviluppo e ricerca su questi temi.
    In particolare, vengono presentate le principali conclusioni emerse dallo studio e del lavoro svolto, fornendo una visione d'insieme delle tecnologie e delle loro implicazioni.

\end{enumerate}

Qualora, al termine dell'analisi, lo studente disponga ancora di tempo a sua disposizione, potrà dedicarsi a approfondimenti o implementazioni aggiuntive, concordate con l'azienda ospitante.