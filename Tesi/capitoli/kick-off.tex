\chapter{Descrizione dello stage}\label{cap:descrizione-stage}

\intro{In questa sezione viene presentata un'analisi del percorso di stage, una descrizione dell'idea e delle tecnologie,
individuando possibili rischi e problematiche che potrebbero presentarsi durante lo svolgimento dello stesso.
Inoltre, viene fornita una panoramica degli obiettivi da raggiungere e della pianificazione delle ore di lavoro.}

\section{Introduzione al progetto e idea dello stage}\label{sec:introduzione-progetto}

Oggi più che mai è importante garantire la sicurezza dei propri dati e delle proprie informazioni, permettendo uno scambio sicuro
e protetto di dati sensibili tra due o più entità. Inoltre, è necessario garantire che le informazioni siano autentiche e non modificate,
per evitare che un utente malintenzionato possa alterare i dati e compromettere la sicurezza del sistema.
Sulla base di questa premessa, in accordo con l'azienda, si è deciso di sviluppare un progetto che permetta di esplorare 
le possibili applicazioni della tecnologia \glsfirstoccur{\gls{blockchaing}} all'interno di un contesto reale e possibilmente applicabile in ambito aziendale. \\

In particolare, la piattaforma oggetto dello sviluppo durante il tirocinio ha come obiettivo principale il controllo dell'accesso ai contenuti
riservati, nel contesto dei film vietati ai minori. Grazie all'utilizzo della tecnologia \glsfirstoccur{\gls{blockchaing}} e dello standard \glsfirstoccur{\gls{w3cg}} \glsfirstoccur{\gls{ssig}},
permettendo quindi di riconoscere i propri utenti evitando condivisione di informazioni, ma criptando i propri dati personali e verificando l'autenticità
dei dati trasmessi. Il progetto si pone quindi di risolvere il problema della condivisione di contenuti riservati, tramite l'utilizzo di tecnologie innovative
e la collaborazione con lo stagista Alessio De Biasi, utilizzando il codice di un progetto di un laureando magistrale presso Computer Science
dell'Università Ca' Foscari richiamato come libreria e di altri stagisti di laurea triennale con progetti similari basati sull'applicazione di tecnologie immutabili e decentralizzate. \\

Il suo utilizzo è molto simile ad altre piattaforme basate sulla condivisione e prenotazione di contenuti multimediali legati ai film:
l'utente si registra presso la piattaforma, fornendo i propri dati personali e la propria età. In questa fase, si ha il riconoscimento dell'utente
tramite l'utilizzo di un \glsfirstoccur{\gls{didg}} e la creazione di un \glsfirstoccur{\gls{vcg}} contenente i dati personali dell'utente, che verrà poi utilizzato per l'autenticazione
e la verifica dei dati. Una volta effettuata la registrazione, l'utente può accedere alla piattaforma e visualizzare i contenuti disponibili, permettendo l'accesso
anche a contenuti limitati qualora l'utente sia maggiorenne. Riconosciuta l'età dell'utente e verificata dalla piattaforma blockchain la sua autenticità,
l'utente può visualizzare il contenuto selezionato dimostrando tramite \glsfirstoccur{\gls{zkpg}} di essere maggiorenne, senza dover fornire ulteriori informazioni personali,
completando la prenotazione. \\

Nello specifico, il progetto è sviluppato da autodidatta, chiedendo consigli e supporto allo stagista De Biasi per confronti e chiarimenti sull'implementazione degli standard in oggetto
e per l'utilizzo del codice dello \textit{smart contract} da lui fornito, usato come libreria per l'implementazione della piattaforma in oggetto.
Inoltre, consigli e supporto sono stati forniti anche in modo reciproco con lo stagista Raffaele Bussolotto, studente presso Ingegneria Informatica di Padova,
con un percorso simile al mio a livello di argomenti, basato sempre sulle tecnologie in oggetto ma con diverso scopo. 
Il loro supporto è stato puramente di confronto, in quanto l'implementazione dei rispettivi progetti è stata fatta in modo indipendente e senza collaborazione diretta. \\
Sulla base di queste premesse nasce il progetto \textbf{VerifiedMovies}.

\section{Analisi preventiva dei rischi}\label{sec:analisi-rischi}

Durante la fase di analisi iniziale sono stati individuati alcuni possibili rischi 
a cui si potrà andare incontro. Si è quindi proceduto a elaborare delle possibili soluzioni per far fronte a tali rischi. \\

\begin{risk}{Inesperienza tecnologica e metodologica} 
    \newline
    \riskdescription{il progetto prevede l'utilizzo di tecnologie e metodologie di cui non si ha piena esperienza e conoscenza, 
    rendendo più difficoltosa la comprensione e l'applicazione delle stesse in fase di implementazione}
    \risksolution{ è stato previsto un periodo di formazione iniziale per studiare le tecnologie e le metodologie da utilizzare
    e l'aiuto/supporto del tutor aziendale e di altri stagisti nella risoluzione di problemi e nella discussione di possibili soluzioni}\label{risk:inesperienza-tecnologica} 
\end{risk}

\medskip

\begin{risk}{Difficoltà di integrazione con lo smart contract da usare come libreria}
    \newline
    \riskdescription{il sistema potrebbe incontrare difficoltà nell'integrazione con lo smart contract da richiamare per la gestione delle identità digitali,
    a causa di errori di programmazione o di problemi di comunicazione tra le parti coinvolte}
    \risksolution{è stato previsto un periodo di test e debug per verificare il corretto funzionamento del sistema e per risolvere eventuali problemi riscontrati,
    approfondendo il dialogo con il tutor aziendale e con lo stagista che ha sviluppato lo smart contract che dovrà essere richiamato in fase di implementazione}\label{risk:integrazione-smart-contract}
\end{risk}

\medskip

\begin{risk}{Ritardi nello sviluppo}
    \newline
    \riskdescription{potrebbero verificarsi ritardi nello sviluppo del sistema a causa di problemi tecnici o imprevisti, come la mancanza di risorse necessarie o la complessità del progetto,
    rimodulando adeguatamente attività e tempistiche}\label{risk:ritardi-sviluppo}
    \risksolution{è stato previsto un periodo di test e debug per verificare il corretto funzionamento del sistema e per risolvere eventuali problemi riscontrati,
    intensificando il dialogo interno con il proponente e gli stagisti per discutere di possibili soluzioni e rimodulando le attività e le tempistiche in base alle necessità}
\end{risk}

\medskip

\begin{risk}{Problemi di sicurezza}
    \newline
    \riskdescription{il sistema potrebbe riscontrare problemi di sicurezza e vulnerabilità a attacchi informatici o la mancanza di protezione dei dati sensibili}
    \risksolution{fin dall'inizio dell'attività di sviluppo, si cerca di garantire un'implementazione al passo con le \textit{best practices} previste dai linguaggi di programmazione utilizzate e lato web, 
    come l'utilizzo di librerie e framework aggiornati e sicuri, la validazione dei dati in input e la protezione da attacchi di tipo \textit{SQL injection} e \textit{XSS}. Inoltre, è possibile prevedere un'attività di testing approfondito,
    ad esempio utilizzando strumenti di analisi statica del codice e di penetration testing, per verificare la corretta implementazione delle misure di sicurezza e la presenza di eventuali vulnerabilità}\label{risk:problemi-sicurezza}
\end{risk}

\clearpage

\begin{risk}{Cambiamenti dei requisiti durante lo sviluppo}
    \newline
    \riskdescription{i requisiti del sistema potrebbero cambiare durante l'attività di implementazione, ad esempio a causa di una modifica delle esigenze del progetto o di un errore di analisi iniziale}
    \risksolution{è possibile prevedere un'attività di pianificazione flessibile, ad esempio utilizzando metodologie agili, come \glsfirstoccur{\gls{scrumg}}, che prevedono una pianificazione adattiva ai cambiamenti dei requisiti. 
    Inoltre, potrebbe essere utile prevedere una comunicazione costante con il tutor aziendale e gli altri stagisti in fase di validazione dei requisiti aggiornati, procedendo con lo sviluppo in modo coerente ed organizzato}\label{risk:cambiamenti-requisiti}
\end{risk}

\section{Obiettivi e requisiti}\label{sec:obiettivi-requisiti}

Il tirocinio prevede lo svolgimento dei seguenti obiettivi, riportando questa notazione, come dal documento \textit{Piano di Lavoro}:
\begin{itemize}
    \item O per i requisiti obbligatori, vincolanti in quanto obiettivo primario richiesto dal committente;
    \item D per i requisiti desiderabili, non vincolanti o strettamente necessari, ma dal riconoscibile valore aggiunto;
    \item F per i requisiti facoltativi, rappresentanti valore aggiunto non strettamente competitivo.
\end{itemize}
Le sigle precedentemente indicate saranno seguite da una coppia sequenziale di numeri, identificativo del requisito.

\begin{itemize}

    \item Obbligatori:
        \begin{itemize}
            \item \underline{\textit{O01}}: descrivere i concetti di base della \glsfirstoccur{\gls{blockchaing}}, tra cui la sua architettura, i nodi della rete, la crittografia e il consenso distribuito;
            \item \underline{\textit{O02}}: analizzare il concetto di \glsfirstoccur{\gls{smartcontractg}} e il linguaggio \glsfirstoccur{\gls{solidityg}}, con particolare attenzione alle vulnerabilità principali e alle tecniche per evitare errori di programmazione;
            \item \underline{\textit{O03}}: approfondire il funzionamento della firma asimmetrica delle transazioni su catena e la validazione dei blocchi, studiando le tipologie di consenso e le catene più conosciute;
            \item \underline{\textit{O04}}: studiare le tecniche di crittografia utilizzate per garantire la sicurezza e la \textit{privacy} delle informazioni personali nell'ambito della \glsfirstoccur{\gls{ssig}} e dei protocolli per la gestione delle identità digitali;
            \item \underline{\textit{O05}}: individuare casi d'uso reali per la \textit{Self Sovereign Identity}, analizzando i consorzi internazionali coinvolti in ambito di ricerca e i finanziamenti europei;
            \item \underline{\textit{O06}}: discutere le sfide e i problemi legati alla \textit{Self Sovereign Identity}, studiando un possibile scenario futuro di applicabilità e basato su \glsfirstoccur{\gls{zkpg}}.
        \end{itemize}

    \item Desiderabili:
        \begin{itemize}
            \item \underline{\textit{D01}}: implementare una \glsfirstoccur{\gls{dappg}} (o dApp) tramite le librerie \glsfirstoccur{\gls{ethersjsg}} oppure \glsfirstoccur{\gls{web3jsg}}, utilizzando uno smart contract di esempio;
            \item \underline{\textit{D02}}: realizzare una UI (interfaccia utente) per la dApp utilizzando \textit{HTML}, \textit{CSS} e \textit{JavaScript};
            \item \underline{\textit{D03}}: utilizzare la \textit{Self Sovereign Identity} e i protocolli studiati per implementare funzionalità di autenticazione utente nella dApp;
            \item \underline{\textit{D04}}: testare e debuggare l'applicazione implementata;
            \item \underline{\textit{D05}}: discutere problemi e sfide relativi all'implementazione di applicazioni decentralizzate su \textit{blockchain} \glsfirstoccur{\gls{ethereumg}}, con particolare attenzione alla scalabilità e alla sicurezza.
        \end{itemize}

    \item Facoltativi:
        \begin{itemize}
            \item \underline{\textit{F01}}: approfondire l'utilizzo di altri linguaggi di programmazione per gli \textit{smart contract}, come \textit{Vyper};
            \item \underline{\textit{F02}}: analizzare l'utilizzo di altre tecnologie \textit{blockchain}, come \textit{Polkadot} o \textit{Cardano}, per implementare la \textit{Self Sovereign Identity};
            \item \underline{\textit{F03}}: esplorare altre funzionalità delle librerie \textit{Ethers.js} e/o \textit{Web3.js}, come l'invio di transazioni;
            \item \underline{\textit{F04}}: investigare i limiti e le sfide legate all'utilizzo della tecnologia \textit{blockchain}.
        \end{itemize}
    
\end{itemize}