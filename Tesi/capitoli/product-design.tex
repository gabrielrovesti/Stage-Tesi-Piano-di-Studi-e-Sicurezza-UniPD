\chapter{Verifica e validazione}
\label{cap:verifica-validazione}

\section{Accessibilità}\label{sec:verifica-validazione-accessibilita}

Al fine di rendere il sito più accessibile possibile, si è deciso di seguire le linee guida fornite da W3C, in particolare quelle riguardanti l'accessibilità dei contenuti web (WCAG) (\cite{site:wcag}), mantenendo il livello AA.
Questo ha comportato l'utilizzo di attributi \textit{alt} per le immagini di contenuto (lasciandolo vuoto per le immagini di presentazione) e l'utilizzo di attributi HTML semantici per la struttura del sito, come 
ad esempio \textit{<header>}, \textit{<main>}, \textit{<footer>} e \textit{<nav>} per gli elementi di navigazione, \textit{label} negli elementi dei form, \textit{<article>} per i contenuti principali della pagina.
Si riporta inoltre l'utilizzo degli attributi \textit{lang} per specificare la lingua del contenuto qualora fosse diversa dall'italiano. \\

Come descritto per la parte dei componenti, è stato creato un componente apposito per aiuto agli \textit{screen reader}, in grado di fornire un link per saltare la navigazione e accedere direttamente al contenuto principale della pagina
oppure l'opzione per tornare su in pagine grandi.
Il sito utilizza un layout a tre pannelli, cercando di rendere fluida l'esperienza di navigazione, evitando lo scorrimento laterale nelle condizioni di visualizzazione più comuni.
La struttura cerca di seguire un livello di intestazioni logiche, utilizzando \textit{<h1>} per il titolo principale della pagina, \textit{<h2>} per i titoli di sezione e \textit{<h3>} per i titoli di sottosezione,
tale da non saltare livelli di intestazione e consentire una navigazione più fluida nelle varie condizioni.
Il carattere utilizzato è \textit{Calibri}, un font \textit{sans-serif} che risulta essere mediamente ben leggibile, adottando convenzionalmente un'interlinea di 1.5 \textit{em} per migliorare la leggibilità del testo. \\

In merito ai contrasti utilizzati, le linee guida richiedono un rapporto di contrasto almeno pari a 4.5:1 per il testo normale e 3:1 per il testo grande, con alcune eccezioni per il testo in grassetto e per il testo non essenziale.
Con l'aiuto del sito \textit{WCAG Contrast Checker} (\cite{site:wcagcontrastchecker}) si è cercato di mantenere un livello di contrasto adeguato nelle varie situazioni, considerando lo sfondo, il testo e lo stato dei link di navigazione (visitato/non visitato) e dei bottoni (\textit{hover}/\textit{focus}).

In dettaglio, possiamo precisare:
\begin{itemize}
    \item per i contrasti nella pagina principale e per il testo nelle varie pagine, un colore con sfumatura \textit{\#ffffff80}, correttamente visibile rispetto all'immagine di sfondo;
    \item per i link di navigazione, un colore con sfumatura \textit{\#ffffff} e per lo stato di hover pari a \textit{\#ff8a8a} con il colore di visitato con sfumatura \textit{\#ffec80}, pienamente contrastanti con la barra di navigazione di colore \textit{141414}, usato qui e nel \textit{footer};
    \item per tutti i bottoni, il colore \textit{4c78a8}, per lo stato di hover il colore \textit{\#303f9f} e per lo stato di focus il colore \textit{\#3f51b580}, contrastanti con lo sfondo \textit{\#ffffff} delle finestre in cui sono usati.
\end{itemize}

Le convenzioni interne del sito, a livello grafico e di navigazione, sono state mantenute coerenti in tutte le pagine, cercando di rendere l'esperienza di navigazione più fluida possibile.
Ogni azione nel sito ha un apposito messaggio di \textit{feedforward} che indirizza l'utente sulle azioni che può compiere e sulle conseguenze che queste hanno, ricevendo un apposito \textit{feedback}.
Nella pagina con più contenuti è stata offerta un'opzione di ricerca per trovare più facilmente i contenuti desiderati, con un apposito messaggio di errore nel caso in cui non vengano trovati risultati.
A questo proposito, è stato utilizzato un componente apposito per la gestione degli errori, che permette di visualizzare un messaggio di errore e di tornare alla pagina precedente. \\

Le immagini utilizzate sono state ottimizzate in dimensione e peso, cercando di mantenere un livello di qualità adeguato, senza appesantire troppo il sito e il suo caricamento.
Le stesse ancore di navigazione hanno nomi esplicativi e sono state posizionate in modo da essere facilmente raggiungibili e cliccabili dall'apposita barra di navigazione.
Inoltre, si è deciso di evitare l'uso di tabelle, che possono risultare poco accessibili, e di utilizzare un layout a griglia per la disposizione dei contenuti grafici, offrendo come descritto in sezione \ref{sec:codifica-front-end} dei punti di rottura per ogni pagina per garantire la fruizione del sito anche su dispositivi mobili.
Un controllo generale di quanto descritto è stato effettuato con l'aiuto dell'estensione \textit{WAVE} (\cite{site:wave}), che ha permesso di individuare alcuni errori e avvisi, corretti e poi modificando di conseguenza,
portando il sito ad essere conforme alle linee guida definite. 

\section{Test di unità}\label{sec:verifica-validazione-test}

Qui vengono descritti i test di unità effettuati sul codice, utilizzando il framework \textit{Jest} (\cite{site:jest}), che permette di effettuare test di unità su codice JavaScript e TypeScript.
Il codice identificativo dei test è strutturato da:
\begin{center}
    \textbf{TU[Numero]}
  \end{center}
\textbf{}
avendo come legenda:
\begin{itemize}
\item \textbf{TU}, cioè il test di unità, in grado di testare una singola unità di codice e di verificare che il suo comportamento sia corretto;
\item \textbf{Numero}, come numero identificativo univoco e progressivo del test in questione.
\end{itemize}

\section{Test di integrazione}\label{sec:verifica-validazione-integrazione}

In questa sezione sono introdotti i test di integrazione, cioè la verifica del corretto funzionamento di più componenti del sistema, in grado di interagire tra loro.

Il codice identificativo dei test è strutturato da:
\begin{center}
    \textbf{TI[Numero]}
  \end{center}
\textbf{}
avendo come legenda:
\begin{itemize}
\item \textbf{TI}, cioè il singolo test di integrazione;
\item \textbf{Numero}, come numero identificativo univoco e progressivo del test in questione.
\end{itemize}

\section{Test di regressione}\label{sec:verifica-validazione-regressione}

Qui sono introdotti i test di regressione, cioè la verifica del corretto funzionamento del sistema dopo l'introduzione di modifiche, verificando di non aver 
introdotto errori in parti del sistema che precedentemente funzionavano correttamente.

Il codice identificativo dei test è strutturato da:
\begin{center}
    \textbf{TR[Numero]}
  \end{center}
\textbf{}
avendo come legenda:
\begin{itemize}
\item \textbf{TR}, cioè il singolo test di regressione;
\item \textbf{Numero}, come numero identificativo univoco e progressivo del test in questione.
\end{itemize}

\section{Test di accettazione}\label{sec:verifica-validazione-accettazione}

Nella sezione sono introdotti i test di accettazione, cioè la verifica del corretto funzionamento del sistema, in grado di soddisfare i requisiti definiti nel capitolo \ref{cap:analisi-requisiti}.

Il codice identificativo dei test è strutturato da:
\begin{center}
    \textbf{TA[Numero]}
  \end{center}
\textbf{}
avendo come legenda:
\begin{itemize}
\item \textbf{TI}, cioè il singolo test di accettazione;
\item \textbf{Numero}, come numero identificativo univoco e progressivo del test in questione.
\end{itemize}

