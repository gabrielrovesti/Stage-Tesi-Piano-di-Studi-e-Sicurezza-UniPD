\chapter{Verifica e validazione}\label{cap:verifica-validazione}
% chktex-file 1
% chktex-file 44

\intro{In questo capitolo verranno riportate le attività di verifica e validazione svolte durante lo sviluppo del progetto,
dettagliando i test svolti a tutti i livelli, dai test di unità fino alla fase di collaudo ed accettazione del prodotto}

\section{Accessibilità}\label{sec:verifica-validazione-accessibilita}

Al fine di rendere il sito più accessibile possibile, si è deciso di seguire le linee guida fornite da \glsfirstoccur{\gls{w3cg}}, in particolare quelle riguardanti l'accessibilità dei contenuti web (WCAG) (\cite{site:wcag}), mantenendo il livello AA.\@
Questo ha comportato l'utilizzo di attributi \textit{alt} per le immagini di contenuto (lasciandolo vuoto per le immagini di presentazione) e l'utilizzo di attributi HTML semantici per la struttura del sito, come 
ad esempio \textit{<header>}, \textit{<main>}, \textit{<footer>} e \textit{<nav>} per gli elementi di navigazione, \textit{label} negli elementi dei form, \textit{<article>} per i contenuti principali della pagina.
Si riporta inoltre l'utilizzo degli attributi \textit{lang} per specificare la lingua del contenuto qualora fosse diversa dall'italiano. \\

Come descritto per la parte dei componenti, è stato creato un componente apposito per aiuto agli \glsfirstoccur{\gls{screenreaderg}}, in grado di fornire un \textit{link} per saltare la navigazione e accedere direttamente al contenuto principale della pagina
oppure l'opzione per tornare su in pagine grandi.
Il sito utilizza un layout a tre pannelli, cercando di rendere fluida l'esperienza di navigazione, evitando lo scorrimento laterale nelle condizioni di visualizzazione più comuni.
La struttura cerca di seguire un livello di intestazioni logiche, utilizzando \textit{<h1>} per il titolo principale della pagina, \textit{<h2>} per i titoli di sezione e \textit{<h3>} per i titoli di sottosezione,
tale da non saltare livelli di intestazione e consentire una navigazione più fluida nelle varie condizioni.
Il carattere utilizzato è \textit{Calibri}, un font \textit{sans-serif} che risulta essere mediamente ben leggibile, adottando convenzionalmente un'interlinea di 1.5 \textit{em} per migliorare la leggibilità del testo. \\

In merito ai contrasti utilizzati, le linee guida richiedono un rapporto di contrasto almeno pari a 4.5:1 per il testo normale e 3:1 per il testo grande, con alcune eccezioni per il testo in grassetto e per il testo non essenziale.
Con l'aiuto del sito \textit{WCAG Contrast Checker} (\cite{site:wcagcontrastchecker}) si è cercato di mantenere un livello di contrasto adeguato nelle varie situazioni, considerando lo sfondo, il testo e lo stato dei link di navigazione (visitato/non visitato) e dei bottoni (\textit{hover}/\textit{focus}).

In dettaglio, possiamo precisare:
\begin{itemize}
    \item per i contrasti nella pagina principale e per il testo nelle varie pagine, un colore con sfumatura \textit{\#ffffff80}, correttamente visibile rispetto all'immagine di sfondo;
    \item per i link di navigazione, un colore con sfumatura \textit{\#ffffff}, per lo stato di hover pari a \textit{\#ff8a8a}, lo stato di visitato con sfumatura \textit{\#ffec80}, tutti pienamente contrastanti con la barra di navigazione di colore \textit{\#141414}, usato qui e nel \textit{footer};
    \item per tutti i bottoni, il colore \textit{\#4c78a8}, per lo stato di hover il colore \textit{\#303f9f} e per lo stato di focus il colore \textit{\#3f51b580}, contrastanti con lo sfondo \textit{\#ffffff} delle finestre in cui sono usati.
\end{itemize}

Le convenzioni interne del sito, a livello grafico e di navigazione, sono state mantenute coerenti in tutte le pagine, cercando di rendere l'esperienza di navigazione più fluida possibile.
Ogni azione nel sito ha un apposito messaggio di \textit{feedforward} che indirizza l'utente sulle azioni che può compiere e sulle conseguenze che queste hanno, ricevendo un apposito \textit{feedback}.
Nella pagina con più contenuti è stata offerta un'opzione di ricerca per trovare più facilmente i contenuti desiderati, con un apposito messaggio di errore nel caso in cui non vengano trovati risultati.
A questo proposito, è stato utilizzato un componente apposito per la gestione degli errori, che permette di visualizzare un messaggio di errore e di tornare alla pagina precedente. \\

Le immagini utilizzate sono state ottimizzate in dimensione e peso, cercando di mantenere un livello di qualità adeguato, senza appesantire troppo il sito e il suo caricamento.
Le stesse ancore di navigazione hanno nomi esplicativi e sono state posizionate in modo da essere facilmente raggiungibili e cliccabili dall'apposita barra di navigazione.
Inoltre, si è deciso di evitare l'uso di tabelle, che possono risultare poco accessibili, e di utilizzare un layout a griglia per la disposizione dei contenuti grafici, offrendo come descritto in sezione~\ref{subsec:codifica-front-end} dei punti di rottura per ogni pagina per garantire la fruizione del sito anche su dispositivi mobili.
Un controllo generale di quanto descritto è stato effettuato con l'aiuto dell'estensione \textit{WAVE} (\cite{site:wave}), che ha permesso di individuare alcuni errori e avvisi, corretti e poi modificando di conseguenza,
portando il sito ad essere conforme alle linee guida definite. 

\section{Test di unità}\label{sec:verifica-validazione-test}

Qui vengono descritti i test di unità effettuati sul codice, utilizzando il \glsfirstoccur{\gls{frameworkg}} \textit{Jest} (\cite{site:jest}), che permette di effettuare test di unità su codice \textit{JavaScript} e \textit{TypeScript}
e la libreria di test \textit{React Testing Library} (\cite{site:reacttestinglibrary}), che permette di effettuare test di unità su codice \textit{React}.
Il codice identificativo dei test è strutturato da:
\begin{center}
    \textbf{TU[Numero]}
  \end{center}
\textbf{}
avendo come legenda:
\begin{itemize}
\item \textbf{TU}, cioè il test di unità, in grado di testare una singola unità di codice e di verificare che il suo comportamento sia corretto;
\item \textbf{Numero}, come numero identificativo univoco e progressivo del test in questione.
\end{itemize}

Ciascun test riporta le seguenti sigle:
\begin{itemize}
\item \textbf{I}: indica che il test è stato implementato e superato;
\item \textbf{NI}: indica che il test non è stato implementato.
\end{itemize}

Per una migliore comprensione del riferimento alle pagine e alle relativi componenti, con denominazione di riferimento in codifica, si rimanda 
alla sezione~\ref{subsec:codice-front-end}.

\renewcommand{\arraystretch}{1.1} % Adjust the spacing inside the cells

% Redefine column type to center the content
\newcolumntype{C}[1]{>{\centering\arraybackslash}p{#1}}

\begin{center}
\captionof{table}{Tabella di tracciamento dei test di unità}\label{tab:test-unita}
\begin{longtable}{|C{2.0cm}|C{3.0cm}|C{8.0cm}|C{2.0cm}|}
\hline
\textbf{ID} & \textbf{Elemento} & \textbf{Descrizione} & \textbf{Stato} \\
\hline
TU1 & AuthContext & Verificare se il componente AuthContext gestisce correttamente l'autenticazione, salvando lo stato dell'utente e fornendo la logica di autenticazione. & I \\
\hline
TU2 & Footer & Verificare se il componente Footer viene visualizzato e caricato correttamente in tutte le pagine dell'applicazione e se il link di ritorno funziona con \textit{screen reader}. & I \\
\hline
\end{longtable}
\end{center}

\begin{center}
\begin{longtable}{|C{2.0cm}|C{3.0cm}|C{8.0cm}|C{2.0cm}|}
\hline
\multicolumn{4}{|c|}{\textbf{Continuazione della tabella~\ref{tab:test-unita}}} \\
\hline
\textbf{ID} & \textbf{Elemento} & \textbf{Descrizione} & \textbf{Stato} \\
\hline
\endhead
TU3 & NavBar & Verificare se il componente NavBar viene visualizzato e caricato correttamente in tutte le pagine dell'applicazione & I \\
\hline
TU4 & NavBar & Verificare se il componente NavBar permette la visualizzazione dei collegamenti corretti in base allo stato di autenticazione dell'utente & I \\
\hline
TU5 & PrivateRoute & Verificare se il componente PrivateRoute gestisce correttamente le rotte protette dell'applicazione, reindirizzando l'utente alla pagina di login se non è autenticato & I \\
\hline
TU6 & ScreenReaderHelp & Verificare se il componente ScreenReaderHelp è progettato correttamente per fornire supporto agli \textit{screen reader}, rendendo il contenuto accessibile solo agli \textit{screen reader} & I \\
\hline
TU7 & SearchBox & Verificare se il componente SearchBox gestisce correttamente la barra di ricerca dei film, effettuando la ricerca in base al titolo e aggiornando correttamente lo stato della ricerca & I \\
\hline
TU8 & Notification & Verificare se il componente Notification gestisce correttamente la visualizzazione degli errori e dei messaggi di notifica nelle diverse situazioni & I \\
\hline
% Aggiungi altre righe come necessario
\end{longtable}
\end{center}

\section{Test di integrazione}\label{sec:verifica-validazione-integrazione}

In questa sezione sono introdotti i test di integrazione, cioè la verifica del corretto funzionamento di più componenti del sistema, in grado di interagire tra loro.
Questo, in particolare, implica l'analisi del comportamento delle pagine e dei componenti grafici da cui sono composte.

Il codice identificativo dei test è strutturato da:
\begin{center}
    \textbf{TI[Numero]}
  \end{center}
\textbf{}
avendo come legenda:
\begin{itemize}
\item \textbf{TI}, cioè il singolo test di integrazione;
\item \textbf{Numero}, come numero identificativo univoco e progressivo del test in questione.
\end{itemize}

\begin{center}
\captionof{table}{Tabella del tracciamento dei test di integrazione}\label{tab:test-integrazione}
\begin{longtable}{|C{2.0cm}|C{3.0cm}|C{8.0cm}|C{2.0cm}|}
\hline
\textbf{ID} & \textbf{Elemento} & \textbf{Descrizione} & \textbf{Stato} \\
\hline
TI1 & HomePage & La pagina viene testata per verificarne il caricamento grafico e delle sue componenti & I \\
\hline
TI1 & HomePage & La pagina viene testata per verificare se le funzioni di creazione del \textit{DID} e della catena degli \textit{issuer} sono chiamate correttamente all'avvio dell'applicazione & I \\
\hline
\end{longtable}
\end{center}

\clearpage

\begin{center}
\begin{longtable}{|C{2.0cm}|C{3.0cm}|C{8.0cm}|C{2.0cm}|}
\hline
\multicolumn{4}{|c|}{\textbf{Continuazione della tabella~\ref{tab:test-integrazione}}} \\
\hline
\textbf{ID} & \textbf{Elemento} & \textbf{Descrizione} & \textbf{Stato} \\
\hline
\endhead
TI3 & RegisterView & La vista di registrazione viene testata per verificare che venga visualizzata correttamente & I \\
\hline
TI4 & RegisterView & La vista di registrazione viene testata per verificare che venga gestito correttamente il processo di registrazione tramite il meccanismo \textit{challenge-response} & I \\
\hline
TI5 & RegisterView & La vista di registrazione viene testata per verificare che vengano gestiti correttamente i controlli di validazione dei campi e le notifiche di successo o di errore in caso di \textit{email} o \textit{DID} già registrati & I \\
\hline
TI6 & LoginView & La vista di \textit{login} viene testata per verificare che venga visualizzata correttamente & I \\
\hline
TI7 & LoginView & La vista di \textit{login} viene testata per verificare che venga gestito correttamente il processo di accesso tramite \textit{DID} & I \\
\hline
TI8 & LoginView & La vista di \textit{login} viene testata per verificare che vengano gestiti correttamente i controlli di validazione dei campi e le notifiche di errore in caso di \textit{DID} non esistente & I \\
\hline
TI9 & LoginView & La vista di \textit{login} viene testata per verificare che venga aperta correttamente la finestra modale di verifica dell'identità in caso di \textit{DID} esistente & I \\
\hline
TI10 & LoginView & La vista di \textit{login} viene testata per verificare che venga gestito correttamente il processo di verifica dell'identità & I \\
\hline
TI11 & MoviesView & La vista dei film viene testata per verificarne il caricamento grafico e delle sue componenti & I \\
\hline
TI12 & MoviesView & La vista dei film viene testata per verificare l'apertura di un film e la visualizzazione dei dettagli corretti & I \\
\hline
TI13 & MoviesView & La vista dei film viene testata per verificare l'apertura della finestra di verifica dell'età e il corretto funzionamento del processo di verifica & I \\
\hline
TI14 & MoviesView & La vista dei film viene testata per verificare la possibilità di condividere un film e l'apertura della finestra di condivisione & I \\
\hline
TI15 & MoviesView & La vista dei film viene testata per verificare la visualizzazione di un messaggio quando non ci sono film disponibili & I \\
\hline
\end{longtable}
\end{center}

\clearpage

\begin{center}
\begin{longtable}{|C{2.0cm}|C{3.0cm}|C{8.0cm}|C{2.0cm}|}
\hline
\multicolumn{4}{|c|}{\textbf{Continuazione della tabella~\ref{tab:test-integrazione}}} \\
\hline
\textbf{ID} & \textbf{Elemento} & \textbf{Descrizione} & \textbf{Stato} \\
\hline
\endhead
TI16 & MoviesView & La vista dei film viene testata per verificare la visualizzazione del profilo utente se autenticato & I \\
\hline
TI17 & MovieBookingView & La vista di prenotazione dei film viene testata per verificare che venga visualizzata correttamente & I \\
\hline
TI18 & MovieBookingView & La vista di prenotazione dei film viene testata per verificare che vengano inizializzati correttamente gli stati di prenotazione & I \\
\hline
TI19 & MovieBookingView & La vista di prenotazione dei film viene testata per verificare che venga gestito correttamente il processo di prenotazione & I \\
\hline
TI20 & MovieBookingView & La vista di prenotazione dei film viene testata per verificare che venga gestito il processo di visualizzazione del riepilogo della prenotazione in modo corretto & I \\
\hline
TI21 & MovieBookingView & La vista viene testata per verificare che venga gestito correttamente il processo di chiusura del modale di conferma prenotazione & I \\
\hline
TI22 & MovieBookingView & La vista viene testata per verificare che venga gestito correttamente il reindirizzamento alla pagina delle prenotazioni & I \\
\hline
TI23 & MovieBookingView & La vista viene testata per verificare che venga gestito il processo di validazione della prenotazione & I \\
\hline
TI24 & BookingListView & La vista di visualizzazione delle prenotazioni viene testata per verificare che venga renderizzata correttamente & I \\
\hline
TI25 & BookingListView & La vista di visualizzazione delle prenotazioni viene testata per verificare che venga mostrato il messaggio corretto quando non ci sono prenotazioni & I \\
\hline
TI26 & BookingListView & La vista di visualizzazione delle prenotazioni viene testata per verificare che vengano mostrate correttamente le prenotazioni presenti & I \\
\hline
TI27 & ErrorView & La vista di errore viene testata per verificare che venga renderizzata correttamente, riportando l'utente alla pagina principale & I \\
\hline
\end{longtable}
\end{center}

\clearpage

\section{Test di regressione}\label{sec:verifica-validazione-regressione}

Qui sono introdotti i test di regressione, cioè la verifica del corretto funzionamento del sistema dopo l'introduzione di modifiche, verificando di non aver 
introdotto errori in parti del sistema che precedentemente funzionavano correttamente.

Il codice identificativo dei test è strutturato da:
\begin{center}
    \textbf{TR[Numero]}
  \end{center}
\textbf{}
avendo come legenda:
\begin{itemize}
\item \textbf{TR}, cioè il singolo test di regressione;
\item \textbf{Numero}, come numero identificativo univoco e progressivo del test in questione.
\end{itemize}

\begin{center}
\captionof{table}{Tabella del tracciamento dei test di regressione}\label{tab:test-regressione}
\begin{longtable}{|C{2.0cm}|C{3.0cm}|C{8.0cm}|C{2.0cm}|}
\hline
\textbf{ID} & \textbf{Elemento} & \textbf{Descrizione} & \textbf{Stato} \\
\hline
TR1 & Applicazione & Si testa che l'applicazione venga correttamente caricata e visualizzata nelle sue funzionalità sul \textit{browser Google Chrome} & I \\
\hline
TR2 & Applicazione & Si testa che l'applicazione venga correttamente caricata e visualizzata nelle sue funzionalità sul \textit{browser Microsoft Edge} & I \\
\hline
TR3 & Applicazione & Si testa che l'applicazione venga correttamente caricata e visualizzata nelle sue funzionalità sul \textit{browser Safari} & I \\
\hline
TR4 & Applicazione & Si testa che l'applicazione venga correttamente caricata e visualizzata nelle sue funzionalità sul \textit{browser Mozilla Firefox} & I \\
\hline
TR5 & Applicazione & Si testa che l'applicazione venga correttamente caricata e visualizzata nelle sue funzionalità sul \textit{browser Opera} & I \\
\hline
TR6 & MovieBookingView & Si verifica che le modifiche apportate alla vista di prenotazione non abbiano causato malfunzionamenti & I \\
\hline
TR7 & MoviesView & Si verifica che eventuali modifiche apportate al meccanismo di verifica della pagina non causino errori o malfunzionamenti compromettendo l'intera logica applicativa & I \\
\hline
TR8 & Applicazione & Si verifica che se vengono implementate modifiche al sistema di gestione di dati, l'applicazione sia comunque in grado di recuperarli correttamente & I \\
\hline
\end{longtable}
\end{center}

\clearpage

\section{Accettazione e collaudo}\label{sec:verifica-validazione-accettazione}

Settimanalmente, al fine di conoscere lo stato di avanzamento del prodotto, veniva organizzato un incontro interno con
il proponente e gli altri stagisti per discutere i progressi raggiunti e discutere eventuali dubbi o problemi riscontrati ad alto livello.
Ciò ha permesso di monitorare il lavoro e di conoscere lo stato dell'implementazione realizzata, garantendo così una maggiore
trasparenza e una migliore comunicazione tra le parti, permettendo di avere un \textit{feedback} indicativo sulle attività da svolgere.
Alcuni chiarimenti si sono avuti con incontri autonomi con lo stagista magistrale Alessio De Biasi, di cui utilizzo il contratto come codice libreria 
per il funzionamento dell'applicazione. \\

Durante l'ultima settimana di stage, il proponente ha richiesto di effettuare un incontro per discutere del prodotto realizzato e per
mostrare l'applicazione completa e funzionante, dimostrando il raggiungimento degli obiettivi prefissati e delle funzionalità richieste, 
partendo dalle pagine ad alto livello e discutendo dei test e dei tipi di firma digitale e gli standard da me implementati. 