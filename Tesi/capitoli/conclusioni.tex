\chapter{Conclusioni}\label{cap:conclusioni}

\intro{In questo capitolo verranno riportate le conclusioni del lavoro svolto, analizzando i risultati ottenuti e le conoscenze acquisite, 
dando una valutazione personale del lavoro svolto.}

\section{Obiettivi raggiunti e consuntivo finale}\label{sec:conclusioni-obiettivi-consuntivo}

In merito al raggiungimento degli obiettivi prefissati per il tirocinio (in sezione~\ref{sec:obiettivi-requisiti}),
si può confermare la soddisfazione di ciascuno. In particolare prima con lo studio e poi con la realizzazione dell'intero progetto,
i concetti di blockchain e di smart contract sono stati acquisiti e compresi, dimostrando con la pratica la comprensione del
L'identità sovrana di \glsfirstoccur{\gls{ssig}} è stato applicato ed implementato in un contesto reale, con la possibilità di
interagire con un'applicazione web e con un'applicazione mobile, dimostrando la sua applicabilità in diversi ambiti.
Questo ha permesso, tramite lo studio e l'implementazione autonoma descritta di standard di identità e di firma digitale, di 
acquisire conoscenze e competenze in ambiti di sicurezza informatica e di crittografia, che sono stati approfonditi e studiati
in modo mirato nel progetto realizzato, garantendo la copertura dei requisiti obbligatori. \\

Come richiesto dalla sezione degli obiettivi desiderabili presenti nella sezione citata, l'applicazione implementa la libreria 
\glsfirstoccur{\gls{web3jsg}} per l'interazione con la \textit{blockchain}, permettendo di interagire con lo \textit{smart contract} in modo corretto.
Inoltre, l'applicazione è stata completamente testata, come riportato dal capitolo~\ref{cap:verifica-validazione}. 
La stessa creazione dell'applicazione permette di esplorare uno scenario di applicazione dell'ambito della \textit{blockchain} e dell'identità connessa
senza trasmettere informazioni personali, risultando così in un'applicazione che rispetta la \textit{privacy} degli utenti.

\section{Conoscenze acquisite e analisi del lavoro svolto}\label{sec:conclusioni-conoscenze-lavoro}

Il tirocinio svolto ha soddisfatto delle aspettative principalmente da un punto di vista conoscitivo, dato che ho capito le implicazioni del mondo \textit{blockchain} e come questo rappresenti,
se ben applicato, un passo importante da un punto di vista di sicurezza. 
Il punto più rilevante di questo tirocinio è stata certamente l'implementazione di questo progetto sfruttando un codice di un laureando magistrale, ampliando questa applicazione a degli standard \glsfirstoccur{\gls{w3cg}}
di non poca importanza, come \glsfirstoccur{\gls{didg}} e \glsfirstoccur{\gls{vcg}}, sviluppando una parte non completamente normata come \glsfirstoccur{\gls{zkpg}}. \\

\clearpage

L'attività è stata molto impegnativa e, nonostante il supporto di massima presente, la parte analitica e di effettivo sviluppo del progetto, specie della parte effettivamente più difficile come 
\textit{Zero Knowledge Proof} sono risultate estremamente teoriche e senza un vero supporto, se non da un punto di vista di puro ragionamento logico, da parte del laureando magistrale Alessio De Biasi, 
di cui ho dovuto adattare il codice per il mio progetto. Le scelte implementative e la loro realizzazione sono state complesse da gestire e da capire in autonomia, 
ma sono state comunque affrontate con successo, permettendo di acquisire conoscenze e competenze in ambiti di sicurezza informatica e di crittografia decisamente importanti per un percorso di Laurea Triennale come il mio. \\

Di massima, è possibile descrivere le principali conoscenze maturate nell'ambito progettuale:
\begin{itemize}
    \item \textbf{comprensione dell'ambito \textit{blockchain} e sviluppo di \textit{smart contract}}: l'interazione di uno \textit{smart contract} e la sua interazione con un'applicazione web e mobile,
    permettono di comprendere come implementare una struttura dati pubblica ed immutabile, cambiando l'idea stessa di programmazione classica e facendo ben comprendere
    come ogni azione fatta in codice possa avere potenziali implicazioni di sicurezza. Ogni transazione è infatti pubblica e richiede una vera comprensione della struttura sottostante 
    e di come scrivere il codice in modo sicuro e corretto, per evitare perdite di dati o di denaro. In questo ambito, utile certamente l'apprendimento autonomo del linguaggio \glsfirstoccur{\gls{solidityg}} e della libreria \textit{web3.js},
    che permettono di interagire con la \textit{blockchain} in modo semplice e adattando tale logica a quella di un'applicazione realmente utilizzabile;
    \item \textbf{analisi di standard di identità digitale e di protocolli di firma digitale}: l'implementazione di \textit{Decentralized Identifiers (DID)} e \textit{Verifiable Credentials (VC)} ha permesso di comprendere come sia possibile implementare un sistema di identità digitale
    partendo da implementazioni definite e preesistenti, tuttora sporadicamente applicate, ma con un ambito di applicazione molto ampio. Di fatto, in un mondo come quello odierno dove la \textit{privacy} rappresenta una giusta preoccupazione per gli utenti medi e per la stessa informatica,
    tale progetto dimostra come, in modo relativamente semplice, sia possibile certificare che l'utente sia chi dice di essere, senza trasmettere informazioni personali, ma solo una firma digitale, che può essere verificata da chiunque sul meccanismo di catena di fiducia
    creato e descritto. Il meccanismo di riconoscimento dell'autenticità delle informazioni trasmesse dall'utente ha richiesto una creazione personalizzata di quanto esistente in ambito sicurezza e \textit{blockchain}, creando un meccanismo conforme agli standard 
    e di facile utilizzo, data la complessità degli argomenti descritti e da me trattati, che richiedono una conoscenza approfondita di basi crittografiche, di sicurezza e di come implementare questi standard in modo corretto e concreto;
    \item \textbf{studio e analisi di \textit{Zero Knowledge Proof}}: la realizzazione e l'applicazione di \textit{Zero Knowledge Proof} è stata una delle parti più complesse da implementare, data la natura estremamente teorica della stessa e della 
    sua successiva applicazione, che richiede necessariamente, dati gli standard precedentemente descritti, la comprensione di standard di firma digitale spesso non accuratamente documentati e che richiedono di studiare a fondo gli standard di riferimento e una ricerca
    in gran parte teorica, ben maggiore di quella prevista da alcuni ambiti di studio, dato il tempo relativamente limitato di effettiva implementazione conforme agli standard.
    \item \textbf{nuove competenze di programmazione e progettuali}: l'analisi e la progettazione dei dettaglio di un'applicazione web e mobile, con l'implementazione di un {\textbf{\glsfirstoccur{\gls{backendg}}}} e di un {\textbf{\glsfirstoccur{\gls{frontendg}}}} hanno permesso di sviluppare ulteriormente le mie conoscenze apprese durante il corso di Ingegneria
    del Software, normando in modo più efficace le attività di codifica e di realizzazione della documentazione, risolvendo sul campo problemi progettuali ma anche concettuali presenti e sviluppando modifiche in modo agile, separando le responsabilità delle componenti e delle pagine, 
    migliorando così la manutenibilità e la comprensione del codice. 
    L'applicazione del linguaggio di programmazione \textit{TypeScript} unito alla comprensione di vulnerabilità legate al linguaggio \textit{Solidity} risulta utile nell'ulteriore comprensione e applicazione da un punto di vista di sviluppo di applicativi web e mobile, 
    analizzando in dettaglio le caratteristiche offerte da questi linguaggi e librerie e strutturando il codice attraverso \textit{design pattern} di riferimento e architetture utili da un punto di vista logico e di sviluppo attraverso la \textit{Continuous Integration} e la \textit{Continuous Delivery} (\cite{site:cicd}), così implementando modifiche in modo continuativo 
    e strutturato, a calendario e nel corso dell'intero progetto. 
\end{itemize}

\section{Possibili sviluppi futuri e scenari di applicabilità}\label{sec:conclusioni-conoscenze-sviluppi}



\section{Valutazione personale}\label{sec:conclusioni-valutazione}

L'attività di stage si è rivelata molto utile per la crescita personale e professionale, permettendo di acquisire nuove conoscenze
e competenze in ambiti di sicurezza informatica e di crittografia, che sono stati approfonditi e studiati in modo mirato nel progetto realizzato.
Inoltre, l'ambito \textit{blockchain} è stato studiato in modo approfondito, permettendo di esplorare in autonomia un campo con ripercussioni
decisamente interessanti nel mondo dell'informatica, decisamente non visti né praticati in corsi universitari ed è stata un'esperienza assolutamente rilevante. 
Ho potuto in questo ambito confrontarmi su aspetti interessanti che offrono nuove prospettive in ambito di sicurezza e che danno una visione delle tecnologie informatiche più consapevole. \\

Spesso ambiti sconosciuti possono essere rilevanti da un punto di vista professionale, e questo è stato un esempio di come
andare oltre i preconcetti dati da un ambito ampiamente teorico e non ormato, si riveli in realtà un'occasione di crescita
e di apprendimento, richiedendo necessariamente di praticare a fondo uno studio mirato delle attività svolte. 
In questi casi, la passione per il conoscere guida e anticipa la necessità di apprendere, dando un'importante possibilità di conoscere e maturare,
altrimenti non offerta in modo così vicino alle realtà aziendali e future rimanendo in ambito puramente accademico.
L'attività di supporto è comunque stata presente di massima, dando ampia possibilità di sviluppo e organizzazione in modo autonomo, 
rimanendo vicino a me come studente e alle mie necessità. \\

Il progetto per me rappresenta una crescita che matura quanto compreso e visto in questi ultimi anni, permettendo di applicare in modo nuovo 
concetti per la maggior parte sconosciuti e con tecnologie nuove, calandomi in un contesto attuale e futuro, usando delle librerie specifiche e sviluppando un
pensiero critico nell'analisi, affinando ulteriormente una visione d'insieme dei corsi fin qui frequentati nel mio percorso di Laurea Triennale, perfezionando il mio metodo di studio 
ed il mio modo di affrontare un progetto di codifica con ripercussioni attuali e future in modo così intenso, completando quanto realizzato nel corso di Ingegneria del Software
e usando tali metodologie per affrontare e suddividere il lavoro in modo autonomo, partendo da quanto appreso in questi mesi.
Quanto studiato rappresenta un importante partenza per il mio futuro percorso di Laurea Magistrale, adattando già ora un codice oggetto di tesi da parte di un laureando magistrale,
ampliando delle competenze trasversali certamente utili per il mio futuro, accademico e non, ed aggiornandomi su un ambito decisamente attuale e di sicuro interesse per possibili sviluppi futuri.