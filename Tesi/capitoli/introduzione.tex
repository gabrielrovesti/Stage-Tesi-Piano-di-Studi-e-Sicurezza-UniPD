\chapter{Introduzione}\label{cap:introduzione}

Introduzione al contesto applicativo.

\noindent Esempio di utilizzo di un termine nel glossario 
\gls{api}. 

\noindent Esempio di citazione in linea 
\cite{site:agile-manifesto}. 

\noindent Esempio di citazione nel pie' di pagina
citazione\footcite{womak:lean-thinking}

\section{L'azienda}

Sync Lab è un'azienda italiana che si occupa di sviluppo software nata nel 2002 con sede principale a Napoli,
rapidamente affermata come System Integrator secondo un processo di ricerca e maturazione di competenze tecnologiche sempre diverse.\\
L'azienda opera in diversi settori, tra cui quello del mobile, della cybersecutrity, del web e della videosorveglianza. \\
L'azienda è in continua crescita e attualmente conta oltre 250 dipendenti, dislocati nelle altre sedi presenti: Roma, Milano, Padova e Verona.

\section{Introduzione al progetto}

Introduzione all'idea dello stage.

Riguardo la stesura del testo, relativamente al documento sono state adottate le seguenti convenzioni tipografiche:
\begin{itemize}
	\item gli acronimi, le abbreviazioni e i termini ambigui o di uso non comune menzionati vengono definiti nel glossario, situato alla fine del presente documento;
	\item per la prima occorrenza dei termini riportati nel glossario viene utilizzata la seguente nomenclatura: \emph{parola}\glsfirstoccur;
	\item i termini in lingua straniera o facenti parti del gergo tecnico sono evidenziati con il carattere \emph{corsivo}.
\end{itemize}

\section{Way of Working e strumenti}

\section{Organizzazione del testo}

\begin{description}
    \item[{\hyperref[cap:tecnologie]{Il secondo capitolo}}] descrive \ldots
    
    \item[{\hyperref[cap:descrizione-stage]{Il terzo capitolo}}] approfondisce \ldots
    
    \item[{\hyperref[cap:analisi-requisiti]{Il quarto capitolo}}] approfondisce \ldots
    
    \item[{\hyperref[cap:progettazione-codifica]{Il quinto capitolo}}] approfondisce \ldots
    
    \item[{\hyperref[cap:verifica-validazione]{Il sesto capitolo}}] approfondisce \ldots
    
    \item[{\hyperref[cap:conclusioni]{Nel settimo capitolo}}] descrive \ldots
\end{description}

