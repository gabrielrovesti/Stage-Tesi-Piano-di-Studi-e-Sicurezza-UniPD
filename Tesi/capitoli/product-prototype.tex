\chapter{Progettazione e codifica}\label{cap:progettazione-codifica}

\intro{In questa sezione, saranno elencate le tecnologie principali utilizzate durante lo sviluppo del sistema oggetto del tirocinio.}

\section{Tecnologie utilizzate}\label{sec:tecnologie-strumenti}

\subsection{React}
React è una libreria JavaScript utilizzata per la creazione di interfacce utente. È stata sviluppata da Facebook e rilasciata nel 2013. 
Essa consente di creare dei componenti riutilizzabili che rappresentano parti dell'interfaccia e gestire lo stato dell'applicazione in modo efficiente e scalabile.
QUesto lo rende adatto allo sviluppo di applicazioni complesse e dinamiche.

\subsection{TypeScript}
TypeScript è un linguaggio di programmazione open source sviluppato da Microsoft. È un super-set di JavaScript che aggiunge tipi statici opzionali al linguaggio,
permettendo di scrivere codice più robusto e manutenibile, grazie alla possibilità di definire interfacce e classi.

\subsection{Solidity}
Solidity è un linguaggio di programmazione orientato agli oggetti per la scrittura di smart contract. È stato sviluppato da Ethereum e permette di gestire
lo stato di un contratto, definire funzioni e interagire con altri contratti all'interno delle blockchain,
grazie alla gestione del sistema di transazioni basato sugli eventi e alla possibilità di definire interfacce.

\subsection{Node.js}
Node.js è un ambiente di runtime open source per l'esecuzione di codice JavaScript lato server. È basato sul motore JavaScript V8 di Google Chrome e permette di gestire
le dipendenze dell'applicazione, grazie al suo package manager \textit{npm}, differenziando le librerie dell'applicaizone e quelle di terze parti.
Tramite semplici comandi, è possibile installare e rimuovere le dipendenze, aggiornarle e gestire le versioni.

\subssection{web3.js}
web3.js è una collezione di librerie JavaScript per poter interagire facilmente con le blockchain, in particolare con Ethereum.
Essa permette di connettersi ad un nodo della blockchain, inviare transazioni e interagire con gli smart contract, gestendo facilmente il collegamento
e l'interazione tra la parte grafica e gli smart contract, definendo modularmente le funzionalità presenti.

\section{Ciclo di vita del software}\label{sec:ciclo-vita-software}

\section{Progettazione}\label{sec:progettazione}

\subsubsection{Namespace 1} %**************************
Descrizione namespace 1.

\begin{namespacedesc}
    \classdesc{Classe 1}{Descrizione classe 1}
    \classdesc{Classe 2}{Descrizione classe 2}
\end{namespacedesc}


\section{Design Pattern Utilizzati}

\section{Codifica}
