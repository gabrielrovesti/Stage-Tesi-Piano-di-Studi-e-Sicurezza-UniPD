% Variables
\newcommand{\myName}{Gabriel Rovesti}
\newcommand{\myTitle}{Titolo della tesi}
\newcommand{\myDegree}{Tesi di Laurea Triennale}
\newcommand{\myUni}{Università degli Studi di Padova}
\newcommand{\myFaculty}{Corso di Laurea in Informatica}
\newcommand{\myDepartment}{Dipartimento di Matematica ``Tullio Levi-Civita''}
\newcommand{\profTitle}{Prof.ssa}
\newcommand{\myProf}{Ombretta Gaggi}
\newcommand{\myLocation}{Padova}
\newcommand{\myAA}{2022-2023}
\newcommand{\myTime}{Maggio 2023}

% PDF/A filecontents
\RequirePackage{filecontents}
\begin{filecontents*}{\jobname.xmpdata}
  \Title{Document's title}
  \Author{Gabriel Rovesti}
  \Language{it-IT}
  \Subject{The abstract, or short description.}
  \Keywords{keyword1\sep keyword2\sep keyword3}
\end{filecontents*}

% Page format settings
% see: http://wwwcdf.pd.infn.it/AppuntiLinux/a2547.htm
\setlength{\parindent}{14pt}    % first row indentation
\setlength{\parskip}{0pt}       % paragraphs spacing
\onehalfspacing                 % line spacing

\geometry{
    left=2cm,
    right=2cm,
    top=2cm,
    bottom=2cm,
}

% Glossary entries
\newglossaryentry{apig} {
    name=\glslink{apig}{Application Program Interface},
    text=Application Program Interface,
    sort=api,
    description={In informatica con il termine \emph{Application Programming Interface API} 
    (ing.\ interfaccia di programmazione di un'applicazione) si indica ogni insieme di procedure disponibili al programmatore, 
    che consente a diverse applicazioni di avere un set definito di regole e protocolli comuni per l'interazione e lo scambio di dati.
    Queste possono essere utilizzate per semplificare l'integrazione complessiva, permettendo un'astrazione dei livelli sottostanti}
}

\newglossaryentry{agileg}{
    name=\glslink{agileg}{Agile},
    text=Agile,
    sort=agile,
    description={In ingegneria del software, 
    con il termine \emph{Agile} si indica un insieme di metodi di sviluppo del software emersi 
    a partire dai primi anni 2000 e fondati su un insieme di principi comuni,
    direttamente o indirettamente derivati dai principi del \emph{Manifesto per lo sviluppo agile del software} 
    (ing. \emph{Manifesto for Agile Software Development}). 
    L'approccio di sviluppo viene definito iterativo e incrementale, prevedendo la suddivisione del progetto in piccole attività
    chiamate \textit{sprint}, della durata di una o due settimane, al termine delle quali viene presentato un incremento del prodotto finale.
    Questa si basa su un'interazione costate e flessibile, basato più sugli individui e le interazioni che sugli strumenti e i processi}
}

\newglossaryentry{scrumg}{
    name=\glslink{scrumg}{Scrum},
    text=Scrum,
    sort=scrum,
    description={In ingegneria del software, \emph{Scrum} è una metodologia di sviluppo 
    iterativa ed incrementale per la gestione del ciclo di sviluppo del software, 
    iterativa in quanto il lavoro viene suddiviso in blocchi (sprint) e
    incrementale perché il lavoro viene suddiviso in parti che vengono consegnate in modo incrementale. 
    In particolare, ciascuna attività prevede un coinvolgimento delle parti molto spesso, definendo periodi
    di retrospettiva e di confronto, per valutare il lavoro svolto e le eventuali modifiche da apportare.
    Il termine \emph{scrum} deriva dal rugby, dove indica una formazione composta dalla linea
    dei giocatori che si fronteggiano e si accaparrano il pallone con le gambe, cercando di spingerlo verso la meta avversaria}
}

\newglossaryentry{gitg}{
    name=\glslink{gitg}{Git},
    text=Git,
    sort=git,
    description={\emph{Git} è un software di controllo versione distribuito utilizzabile da interfaccia a riga di comando, 
    creato da Linus Torvalds nel 2005. 
    Lo scopo di Git è quello di gestire progetti con velocità e semplicità, 
    garantendo allo stesso tempo la possibilità di gestire flussi di lavoro complessi 
    sulla base di un sistema di controllo di versione non lineare e distribuito. 
    Git permette di tenere traccia di tutte le modifiche apportate al codice sorgente 
    di un progetto sviluppato da più persone e di coordinarle}
}

\newglossaryentry{softwareg}{
    name=\glslink{softwareg}{Software},
    text=software,
    sort=software,
    description={Insieme di programmi, dati e documentazione che compongo un sistema o un progetto informatico.
    Esso viene progettato, sviluppato e mantenuto da un gruppo di persone, chiamate sviluppatori, al fine di soddisfare le esigenze del prodotto 
    portando una soluzione ad uno specifico problema nel dominio applicativo di interesse. Questo può essere sviluppato 
    seguendo varie metodologie, tra cui ad esempio \emph{Agile} o \emph{Scrum}}
}

\newglossaryentry{umlg} {
    name=\glslink{umlg}{Unified Modeling Language},
    text=UML,
    sort=uml,
    description={In ingegneria del software \emph{UML, Unified Modeling Language} (ing.\ linguaggio di modellazione unificato) è un linguaggio di modellazione utilizzato per Descrivere
    e progettare sistemi complessi. Questo fornisce un insieme di notazioni grafiche atte a definire il sistema e i suoi comportamenti, comprendendo ogni parte di interazione e comunicazione
    tra le varie parti. Esso viene utilizzato nelle fasi di analisi dei requisiti, progettazione, implementazione e test del prodotto software.
    Esso è normalmente composto da diversi tipi di diagrammi con diversi significati, quali diagrammi delle classi o dei casi d'uso}
}

\newglossaryentry{ideg}{
    name=\glslink{ideg}{Integrated Development Environment},
    text=IDE,
    sort=ide,
    description={In informatica con il termine \emph{Integrated Development Environment} (ing.\ ambiente di sviluppo integrato) si indica un software che, in fase di programmazione, supporta i programmatori nello sviluppo del codice sorgente di un programma. 
    Solitamente un \emph{IDE} è composto da un editor di codice sorgente, un compilatore ed un debugger. 
    Inoltre, spesso, fornisce strumenti per l'automazione di alcune operazioni ripetitive, 
    per la navigazione all'interno del codice e per semplificare alcune operazioni di sviluppo}
}

\newglossaryentry{blockchaing}{
    name=\glslink{blockchaing}{Blockchain},
    text=blockchain,
    sort=blockchain,
    description={In informatica con il termine \emph{Blockchain} si indica una struttura dati condivisa e immutabile.
    La blockchain è resa immutabile dall'utilizzo di funzioni crittografiche di \textit{hash} e dalla struttura dati a blocchi concatenati.
    ed è condivisa in quanto è distribuita in una rete \textit{peer-to-peer}. Essa è costituita da una catena di blocchi, in cui ciascuno contiene un insieme di transazioni o di informazioni, 
    che viene validato e aggiunto alla catena da un processo di consenso tra i nodi della rete}
}

\newglossaryentry{pocg}{
    name=\glslink{pocg}{Proof of Concept}
    text=Proof of Concept,
    sort=proof of concept,
    description={In ingegneria del software con il termine \emph{Proof of Concept} (ing.\ prova di fattibilità) si indica un'implementazione o esperimento che dimostra la fattibilità di un concetto o di un'idea. 
    Il \emph{Proof of Concept} è generalmente focalizzato su singoli aspetti o caratteristiche del progetto, 
    dimostrando che il concetto o l'idea è praticabile e funzionante}
}

\newglossaryentry{w3cg}{
    name=\glslink{w3cg}{World Wide Web Consortium},
    text=W3C,
    sort=w3c,
    description={Il \emph{World Wide Web Consortium} (W3C) è un'organizzazione internazionale che ha come scopo quello di sviluppare tutte le potenzialità del World Wide Web. 
    Il W3C produce e promuove standard tecnici aperti e liberi per il Web, allo scopo di garantirne la crescita e l'innovazione attraverso l'implementazione di nuove tecnologie e standard.
    Tra gli standard più conosciuti si possono citare \textit{HTML}, \textit{CSS}, il protocollo \textit{HTTP}, le linee guide \textit{WCAG} per l'accessibilità e, nel contesto della tesi, lo standard \textit{Verifiable Credentials} e \textit{Decentralized Identifiers}}
}

\newglossaryentry{ssig}{
    name=\glslink{ssig}{Self Sovereign Identity},
    text=Self Sovereign Identity,
    sort=ssi,
    description={In informatica con il termine \emph{Self-Sovereign Identity (SSI)} (ing.\ identità autonoma) 
    modello di identità digitale basato sulla proprietà e la gestione dei dati personali da parte dell'utente. 
    In un sistema di identità digitale SSI, l'utente ha il pieno controllo dei propri dati personali 
    e decide quali informazioni condividere e con quali soggetti. L'obiettivo è quello di ridurre la dipendenza da terze parti 
    per la gestione delle identità digitali, aumentando la sicurezza e la privacy degli utenti. 
    L'SSI si basa su tecnologie come la blockchain e la crittografia, che consentono di creare registri distribuiti e sicuri delle 
    informazioni personali, conservate in modo certificato ed immutabile}
}

\newglossaryentry{didg}{
    name=\glslink{didg}{Decentralized Identifier},
    text=Decentralized Identifier,
    sort=decentralized identifier,
    description={In informatica con il termine \emph{Decentralized Identifier (DID)} (ing.\ identificatore decentralizzato) si indica un identificatore univoco, 
    che può essere utilizzato per identificare un'entità digitale, come una persona, un'organizzazione o un dispositivo. 
    Esso è composto da un prefisso che identifica la rete in cui è stato creato, seguito da un identificatore univoco generato dall'utente.
    Questo è basato su tecnologie decentralizzate e distribuite come blockchain, sviluppato come parte dello standard \textit{W3C} Verifiable Credentials,
    al fine di avere un insieme di tecnologie per invio e verifica di credenziali digitali verificabili e sicure}
}

\newglossaryentry{vcg}{
    name=\glslink{vcg}{Verifiable Credentials},
    text=Verifiable Credentials,
    sort=vc,
    description={In informatica con il termine \emph{Verifiable Credential (VC)} (ing.\ credenziale verificabile) si indica un documento digitale che contiene informazioni relative ad un'entità digitale, 
    come una persona, un'organizzazione o un dispositivo. Una \emph{VC} è un documento firmato digitalmente da un'autorità che ne certifica l'autenticità e che può essere verificato da terze parti,
    contenendo le informazioni principali di una persona come nome, cognome, data di nascita, luogo di nascita, ecc. Anche questo è basato su tecnologie decentralizzate e distribuite come blockchain,
    parte dello standard \textit{W3C} omonimo}
}

\newglossaryentry{zkpg}{
    name=\glslink{zkpg}{Zero Knowledge Proof},
    text=Zero Knowledge Proof,
    sort=zkp,
    description={In crittografia con il termine \emph{Zero-Knowledge Proof (ZKP)} (ing.\ prova a conoscenza zero) si indica un protocollo che permette ad un soggetto di dimostrare di conoscere un certo dato, 
    senza doverlo rivelare. Le ZKP sono utilizzate in diverse applicazioni, come la verifica dell'identità digitale, la gestione delle transazioni finanziarie, la condivisione delle informazioni sensibili}
}

\newglossaryentry{ethereumg}{
    name=\glslink{ethereumg}{Ethereum},
    text=Ethereum,
    sort=ethereum,
    description={In informatica con il termine \emph{Ethereum} si indica una piattaforma decentralizzata per la creazione e pubblicazione di applicazioni decentralizzate. 
    La piattaforma è basata su una blockchain pubblica e permette di creare applicazioni decentralizzate che possono essere eseguite in una macchina virtuale
    chiamata \textit{Ethereum Virtual Machine (EVM)}, utilizzando il linguaggio di programmazione \textit{Solidity} basato su \textit{smart contract}, che 
    consentono di automatizzare e garantire la sicurezza delle informazioni trasmesse in modo tracciato e immutabile}
}

\newglossaryentry{ethersjsg}{
    name=\glslink{ethersjsg}{Ethers.js},
    text=Ethers.js,
    sort=ethers.js,
    description={In informatica con il termine \emph{Ethers.js} si indica una libreria JavaScript per interagire con la blockchain di Ethereum. 
    Questa libreria fornisce un'interfaccia semplificata e sicura per la gestione di contratti intelligenti e delle loro transazioni
    secondo le tecnologie di crittografia asimmetrica e la comunicazione tra i nodi della rete}
}

\newglossaryentry{solidityg}{
    name=\glslink{solidityg}{Solidity},
    text=Solidity,
    sort=solidity,
    description={In informatica con il termine \emph{Solidity} si indica un linguaggio di programmazione ad alto livello per la creazione di smart contract. 
    Esso ha una sintassi simile ai linguaggi C++, Java e JavaScript, progettato per fornire un livello di sicurezza e affidabilità elevato per la gestione
    ed esecuzione dei contratti intelligenti sulla blockchain di Ethereum. 
    Esso supporta diverse funzionalità, tra cui la definizione di variabili, l'implementazione di funzioni, la definizione di strutture dati e la gestione degli eventi}
}

\newglossaryentry{web3jsg}{
    name=\glslink{web3jsg}{Web3.js},
    text=Web3.js,
    sort=web3.js,
    description={In informatica con il termine \emph{Web3.js} si indica una libreria JavaScript per interagire principalmente con la blockchain di Ethereum. 
    La libreria permette di creare applicazioni decentralizzate che possono essere eseguite in un browser web.
    Questa fornisce un'API completa per la creazione e la gestione dei contratti intelligenti, la creazione di transazioni, la gestione delle chiavi crittografiche 
    e la lettura dei dati dalla blockchain.\ ethers.js è basata su tecnologie di crittografia asimmetrica e sulla comunicazione con i nodi della blockchain di Ethereum attraverso il protocollo \textit{JSON-RPC}}
}

\newglossaryentry{vpg}{
    name=\glslink{vpg}{Verifiable Presentation},
    text=Verifiable Presentation,
    sort=verifiable presentation,
    description={In informatica con il termine \emph{Verifiable Presentation (VP)} (ing.\ presentazione verificabile) si indica un documento digitale che contiene informazioni relative ad un'entità digitale,
    come una persona, un'organizzazione o un dispositivo. Una \emph{VP} è un documento firmato digitalmente da un'autorità che ne certifica l'autenticità e che può essere verificato da terze parti,
    ciascuna contenente una serie di credenziali con un verificatore specifico, in grado di evidenziare la paternità dei dati dopo un processo di verifica,
    non contenenti informazioni personali, ma solo un identificatore univoco. Anche questo è basato su tecnologie decentralizzate e distribuite come blockchain}
}

\newglossaryentry{dappg}{
    name=\glslink{vpg}{Decentralized Application},
    text=Decentralized Application,
    sort=decentralized application,
    description={In informatica con il termine \emph{Decentralized Application (DApp)} (ing.\ applicazione decentralizzata) si indica un'applicazione che utilizza una blockchain per la gestione dei dati e delle transazioni.
    Queste applicazioni sono eseguite in una rete peer-to-peer, senza un'autorità centrale che ne controlla il funzionamento, ma sono gestite da tutti i nodi della rete}
}

\newglossaryentry{smartcontractg}{
    name=\glslink{smartcontractg}{smart contract},
    text=smart contract,
    sort=smart contract,
    description={In informatica con il termine \emph{Smart Contract} (ing.\ contratto intelligente) si indica un programma informatico che permette di automatizzare e garantire la sicurezza delle informazioni trasmesse in modo tracciato e immutabile.
    Questi sono eseguiti in una blockchain, in particolare in quella di Ethereum, e sono scritti in un linguaggio di programmazione ad alto livello chiamato \textit{Solidity}}
}

\newglossaryentry{frontendg}{
    name=\glslink{frontendg}{front-end},
    text=front-end,
    sort=frontend,
    description={In informatica con il termine \emph{front-end} si indica la parte di un'applicazione che interagisce direttamente con l'utente, 
    fornendo un'interfaccia grafica per l'interazione con l'applicazione stessa oppure con sistemi esterni in grado di produrre input}
}

\newglossaryentry{backendg}{
    name=\glslink{backendg}{back-end},
    text=back-end,
    sort=backend,
    description={In informatica con il termine \emph{back-end} si indica la parte di un'applicazione che interagisce con il database, 
    fornendo un'interfaccia per la gestione dei dati e delle informazioni, permettendo di fornire le effettive interazioni con l'utente finale}
}

\newglossaryentry{repositoryg}{
    name=\glslink{repositoryg}{repository},
    text=repository,
    sort=repository,
    description={In informatica con il termine \emph{repository} si indica un ambiente di un sistema informativo, in cui vengono gestiti i metadati, attraverso tabelle relazionali, 
    e i file, attraverso un file system gerarchico}
}

\newglossaryentry{jsong}{
    name=\glslink{jsong}{JSON},
    text=JSON,
    sort=json,
    description={In informatica con il termine \emph{JSON} si indica un formato di testo per la memorizzazione e lo scambio di dati, basato sul linguaggio di programmazione JavaScript.
    Questo formato è basato su coppie chiave-valore, in cui la chiave è una stringa e il valore può essere un oggetto, un array, un numero, una stringa, un booleano o nullo.
    Tale formato si adatta bene alla memorizzazione di dati strutturati, come ad esempio le informazioni contenute in un database}
}

\newglossaryentry{urig}{
    name=\glslink{urig}{URI},
    text=URI,
    sort=uri,
    description={In informatica con il termine \emph{URI} si indica una sequenza di caratteri che identifica in modo univoco una risorsa. 
    Questo formato è utilizzato per identificare le risorse in Internet, come ad esempio i documenti, le immagini, i video, i servizi, le risorse di rete e altro ancora}
}

\newglossaryentry{screenreaderg}{
    name=\glslink{screenreaderg}{screen reader},
    text=screen reader,
    sort=screen reader,
    description={In informatica con il termine \emph{screen reader} si indica un software che permette di convertire il testo presente sullo schermo in un formato audio, 
    permettendo di usufruire del contenuto presentato attraverso sintesi vocale}
}

\newglossaryentry{frameworkg}{
    name=\glslink{frameworkg}{framework},
    text=framework,
    sort=framework,
    description={In informatica con il termine \emph{framework} si indica un'architettura logica di supporto su cui un software può essere progettato e realizzato, 
    facilitandone lo sviluppo da parte del programmatore. Questo permette di avere un'astrazione del codice, permettendo di concentrarsi sulla logica dell'applicazione, 
    senza dover gestire le parti più complesse}
} % Glossary definitions
\makeglossaries

\bibliography{appendice/bibliografia}

\defbibheading{bibliography} {
    \cleardoublepage
    \phantomsection
    \addcontentsline{toc}{chapter}{\bibname}
    \chapter*{\bibname\markboth{\bibname}{\bibname}}
}

% Spacing between entries
\setlength\bibitemsep{1.5\itemsep}

\titleformat{\chapter}[display]
  {\normalfont\huge\bfseries}{\chaptertitlename\ \thechapter}{20pt}{\Huge}
\titlespacing{\chapter}
  {0pt}{50pt}{20pt}

\DeclareBibliographyCategory{opere}
\DeclareBibliographyCategory{web}

\addtocategory{opere}{womak:lean-thinking}
\addtocategory{web}{site:agile-manifesto}

\defbibheading{opere}{\section*{Riferimenti bibliografici}}
\defbibheading{web}{\section*{Siti Web consultati}}

\captionsetup{
    tableposition=top,
    figureposition=bottom,
    font=small,
    format=hang,
    labelfont=bf
}

% Images path
\graphicspath{{immagini/}}

\hypersetup{
    %hyperfootnotes=false,
    %pdfpagelabels,
    %draft,	% = elimina tutti i link (utile per stampe in bianco e nero)
    colorlinks=true,
    linktocpage=true,
    pdfstartpage=1,
    pdfstartview=,
    % decommenta la riga seguente per avere link in nero (per esempio per la stampa in bianco e nero)
    %colorlinks=false, linktocpage=false, pdfborder={0 0 0}, pdfstartpage=1, pdfstartview=FitV,
    breaklinks=true,
    pdfpagemode=UseNone,
    pageanchor=true,
    pdfpagemode=UseOutlines,
    plainpages=false,
    bookmarksnumbered,
    bookmarksopen=true,
    bookmarksopenlevel=1,
    hypertexnames=true,
    pdfhighlight=/O,
    %nesting=true,
    %frenchlinks,
    urlcolor=webbrown,
    linkcolor=RoyalBlue,
    citecolor=webgreen
    %pagecolor=RoyalBlue,
    %urlcolor=Black, linkcolor=Black, citecolor=Black, %pagecolor=Black,
}

% Itemize symbols
%\renewcommand{\labelitemi}{$\ast$}
%\renewcommand{\labelitemi}{$\bullet$}
%\renewcommand{\labelitemii}{$\cdot$}
%\renewcommand{\labelitemiii}{$\diamond$}
%\renewcommand{\labelitemiv}{$\ast$}

% Listings setup
\lstset{
    language=[LaTeX]Tex,%C++,
    keywordstyle=\color{RoyalBlue}, %\bfseries,
    basicstyle=\small\ttfamily,
    %identifierstyle=\color{NavyBlue},
    commentstyle=\color{Green}\ttfamily,
    stringstyle=\rmfamily,
    numbers=none, %left,%
    numberstyle=\scriptsize, %\tiny
    stepnumber=5,
    numbersep=8pt,
    showstringspaces=false,
    breaklines=true,
    frameround=ftff,
    frame=single
}

\definecolor{webgreen}{rgb}{0,.5,0}
\definecolor{webbrown}{rgb}{.6,0,0}

% \omiss produces '[...]'
\newcommand{\omissis}{[\dots\negthinspace]}

% Hyphenation rules
\hyphenation{
    ma-cro-istru-zio-ne
    gi-ral-din
}

\newcommand{\sectionname}{sezione}
\addto\captionsitalian{\renewcommand{\figurename}{Figura}
                       \renewcommand{\tablename}{Tabella}}

\newcommand{\glsfirstoccur}{\ap{{[g]}}}

\newcommand{\intro}[1]{\emph{\textsf{#1}}}

% Risks environment
\newcounter{riskcounter}                % define a counter
\setcounter{riskcounter}{0}             % set the counter to some initial value

%%%% Parameters
% #1: Title
\newenvironment{risk}[1]{
    \refstepcounter{riskcounter}        % increment counter
    \par \noindent                      % start new paragraph
    \textbf{\arabic{riskcounter}. #1}   % display the title before the content of the environment is displayed
}{
    \par\medskip
}

\newcommand{\riskname}{Rischio}

\newcommand{\riskdescription}[1]{\textbf{\\Descrizione:} #1.}

\newcommand{\risksolution}[1]{\textbf{\\Soluzione:} #1.}

% Use case environment
\newcounter{usecasecounter}             % define a counter
\setcounter{usecasecounter}{0}          % set the counter to some initial value

%%%% Parameters
% #1: ID
% #2: Nome
\newenvironment{usecase}[2]{
    \renewcommand{\theusecasecounter}{\usecasename#1}  % this is where the display of
                                                        % the counter is overwritten/modified
    \refstepcounter{usecasecounter}             % increment counter
    \vspace{10pt}
    \par \noindent                              % start new paragraph
    {\large \textbf{\usecasename#1: #2}}       % display the title before the
                                                % content of the environment is displayed
    \medskip
}{
    \medskip
}

\newcommand{\usecasename}{UC}

\newcommand{\usecaseactors}[1]{\textbf{\\Attori Principali:} #1. \vspace{4pt}}
\newcommand{\usecasepre}[1]{\textbf{\\Precondizioni:} #1. \vspace{4pt}}
\newcommand{\usecasedesc}[1]{\textbf{\\Descrizione:} #1. \vspace{4pt}}
\newcommand{\usecasepost}[1]{\textbf{\\Postcondizioni:} #1. \vspace{4pt}}
\newcommand{\usecasealt}[1]{\textbf{\\Scenario Alternativo:} #1. \vspace{4pt}}

% Namespace description environment
\newenvironment{namespacedesc}{
    \vspace{10pt}
    \par \noindent  % start new paragraph
    \begin{description}
}{
    \end{description}
    \medskip
}

\newcommand{\classdesc}[2]{\item[\textbf{#1:}] #2}
